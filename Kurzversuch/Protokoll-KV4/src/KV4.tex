\documentclass[12pt]{article}

\usepackage{a4wide}
\usepackage{graphics}
\usepackage{amsmath, amssymb, amsthm}
\usepackage{epsfig}
\DeclareGraphicsRule{.tif}{png}{.png}{`convert #1 `dirname #1`/`basename #1 .tif`.png}
\usepackage{caption}
\usepackage{subcaption}
\usepackage[ngerman]{babel}
\usepackage{hyperref}

\title{- KV4 - \\ Zählen einzelner Photonen mit einem Silicon Photomultiplier (SiPM)}
\author{Paul Callsen & Darek Petersen}
\date{25.08.2024}

\begin{document}

\maketitle
\thispagestyle{empty}
\clearpage

\begin{abstract}
Eigentliches Protokoll in Overleaf zu finden.

  Hier steht eine Zusammenfassung, die die Motivation, das Ziel, die Methoden und die Ergebnisse \emph{kurz und pr\"agnant} darstellt.
\end{abstract}

\section{Einleitung}
Aufgabenstellung, Ziel der Messungen
\section{Vorbereitungsfragen}


\section{Theoretische Grundlagen}
Kurze Zusammenfassung der wichtigsten Überlegungen (Formeln), die zum Verständnis des Versuches notwendig sind.\\
Für Details: Hinweise auf die Literatur, z.B.:
Die Radioaktivität wurde erstmals 1896 von Henri Becquerel entdeckt \cite{Becquerel:261888}.


\section{Experimenteller Aufbau und Durchführung}
Prinzip der Messmethode, Einzelheiten der Apparatur (nur soweit sie spezifisch und zur Beurteilung der Qualität der Messergebnisse von Bedeutung sind).

\section{Ergebnisse}
Messergebnisse mit Fehlerdiskussion.\\
Im Protokoll nicht unbedingt alle Messwerte einzeln aufführen, sondern nur die zur Berechnung der Ergebnisse wesentlichen Daten. \\


\subsection{Unterabschnitt}
Wichtig: Immer Verweis auf die Abbildung, auf die Bezug genommen wird.
Z.B.: In Abb.~\ref{meinlabel1} sieht man einen Platzhalter.

\begin{figure}[h!]
  \centering
  \includegraphics[width=0.8\textwidth]{./Platzhalter.jpg}
  \caption{Griffige Bildunterschrift}
  \label{meinlabel1}
\end{figure}

Zwei Abbildungen nebeneinander sind in Abb.~\ref{meinlabel2} dargestellt.
Und so sieht eine Tabelle aus (Tabelle~\ref{tabdummy}).

Weitere Messwerte k\"onnen im Anhang \label{anhang} aufgelistet werden.


\begin{figure}[h!]
  \centering
  \begin{subfigure}{0.49\textwidth}
    \includegraphics[width=\textwidth]{./Platzhalter.jpg}
    \caption{Bildunterschrift}
  \end{subfigure}
  \hfill
  \begin{subfigure}{0.49\textwidth}
    \includegraphics[width=\textwidth]{./Platzhalter.jpg}
    \caption{Bildunterschrift}
  \end{subfigure}
  \caption{Griffige Bildunterschrift}
  \label{meinlabel2}
\end{figure}



\begin{table}[h]
  \begin{center}
    \begin{tabular}{|c|c|c|}
      \hline
      A & B [mm] & C [V] \\
      \hline
      1 & 1.0 & 10.0 \\
      2 & 1.1 & 11.0 \\
      3 & 1.3 & 12.0 \\
      \hline
    \end{tabular}
    \caption{Tabellenunterschrift}
    \label{tabdummy}
  \end{center}
\end{table}%


\section{Diskussion}
Diskussion der physikalischen Ergebnisse.
Vergleich mit den theoretischen Erwartungen, Fehlerdiskussion.


\section{Zusammenfassung}
Wiederholung/Zusammenfassung der wesentlichen Ergebnisse des Versuchs sowie deren Einordnung. Eventuell Ausblick.


\thispagestyle{empty}
\bibliography{FP_protokoll_template}
\bibliographystyle{unsrt}
\addcontentsline{toc}{chapter}{Bibliography}

\clearpage
\appendix
\section{Anhang} \label{anhang}
Hierhin geh\"oren in der Regel Originalmessprotokolle oder weitere Tabellen, die nicht f\"ur das Verst\"andnis des Protokolls essentiell sind.

\end{document}
